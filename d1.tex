\documentclass[12pt]{report}
\usepackage{graphicx}
\usepackage{hyperref}
\usepackage{float}
\begin{document}
	\title{Function arccos(x)}
	\author{Prabhpreet Singh (40068870)}
	
	\maketitle
	\section*{Function arccos(x)}
	
	The arccosine(x) function is the inverse of cosine of x. It returns the angle at which the cosine is x. It is called by the abbreviated form acos(x). $\arccos(x) = \cos-1(x)$
	\subsection*{Domain and Co-Domain of arcos(x)}
	\begin{equation}
		y = \arccos(x)
	\end{equation}
	\subsection*{Domain:}
	$-1\leq x \leq1$
	
	\subsection*{Range:} 
	$0^\circ \leq y \leq 180^\circ$
	\subsection*{Graph of arccos(x)}
	\begin{figure}[H]
		\centering
		\includegraphics[width=8cm, height=8cm]{arccos-graph.png}
		\label{Graph of arccos(x)}
	\end{figure}
	\subsection*{Characteristics of arccos(x)}
	\begin{itemize}
		\item The value of angle is the highest at -1 and it decreases and becomes zero at +1.
		\item The arccosine(x) is a one-to-one function. Its range is limited to 180° because after that values repeat itself which violates one-to-one property.
		\item The function can be used to calculate the base angle in a triangle by calculating base/hypotenuse to that angle if we know cos-1 of that value.
	\end{itemize}
	\subsection*{References}
	\begin{itemize}
		\item \url{https://www.mathopenref.com/arccos.html}
		\item \url{https://www.rapidtables.com/math/trigonometry/arccos.html}
	\end{itemize}
\end{document}