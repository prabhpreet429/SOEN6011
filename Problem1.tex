\documentclass[12pt]{report}
\usepackage{graphicx} % Used for including images 
\usepackage{hyperref} % Used for including hyperlinks
\usepackage{float} % Used for positioning of image
\begin{document}
	\title{Function arccos(x)} % Main page title
	\author{Name: Prabhpreet Singh \\Student Id: 40068870} % Used for author 
	
	\maketitle
	\section*{Problem 1} % Used for main section 
	\subsection*{Function arccos(x)} % Used for subsection
	The arccosine(x) function is the inverse of cosine of x. It returns the angle at which the cosine is x. It is called by the abbreviated form acos(x). $\arccos(x) = \cos-1(x)$
	\subsection*{Domain and Co-Domain of arcos(x)}
	\begin{equation} % Used to begin equation section
	y = \arccos(x)
	\end{equation}
	Domain: $-1\leq x \leq1$\\Range: $0^\circ \leq y \leq 180^\circ$
	\subsection*{Graph of arccos(x)}
	\begin{figure}[H]% Used for including image along with flow of document
		\centering % Used for positioning of image
		\includegraphics[width=6cm, height=5cm]{arccos-graph.png}
		\label{Graph of arccos(x)} % Used to label the image
	\end{figure}
	\subsection*{Characteristics of arccos(x)}
	\begin{itemize} % Used to list items in bullets
		\item The value of angle is the highest at -1 and it decreases and becomes zero at +1.
		\item The arccosine(x) is a one-to-one function. Its range is limited to 180° because after that values repeat itself which violates one-to-one property.
		\item The function can be used to calculate the base angle in a triangle by calculating base/hypotenuse to that angle if we know cos-1 of that value.
	\end{itemize}
	\subsection*{References}
	\begin{itemize} 
		\item \url{https://www.mathopenref.com/arccos.html}
		\item \url{https://www.rapidtables.com/math/trigonometry/arccos.html}
	\end{itemize}
\end{document}